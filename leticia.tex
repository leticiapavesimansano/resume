%%%%%%%%%%%%%%%%%
% This is an sample CV template created using altacv.cls
% (v1.1.5, 1 December 2018) written by LianTze Lim (liantze@gmail.com). Now compiles with pdfLaTeX, XeLaTeX and LuaLaTeX.
%
%% It may be distributed and/or modified under the
%% conditions of the LaTeX Project Public License, either version 1.3
%% of this license or (at your option) any later version.
%% The latest version of this license is in
%%    http://www.latex-project.org/lppl.txt
%% and version 1.3 or later is part of all distributions of LaTeX
%% version 2003/12/01 or later.
%%%%%%%%%%%%%%%%

%% If you need to pass whatever options to xcolor
\PassOptionsToPackage{dvipsnames}{xcolor}

%% If you are using \orcid or academicons
%% icons, make sure you have the academicons
%% option here, and compile with XeLaTeX
%% or LuaLaTeX.
% \documentclass[10pt,a4paper,academicons]{altacv}

%% Use the "normalphoto" option if you want a normal photo instead of cropped to a circle
% \documentclass[10pt,a4paper,normalphoto]{altacv}

\documentclass[10pt,a4paper,ragged2e]{altacv}

%% AltaCV uses the fontawesome and academicon fonts
%% and packages.
%% See texdoc.net/pkg/fontawecome and http://texdoc.net/pkg/academicons for full list of symbols. You MUST compile with XeLaTeX or LuaLaTeX if you want to use academicons.

% Change the page layout if you need to
\geometry{left=1cm,right=9cm,marginparwidth=6.8cm,marginparsep=1.2cm,top=1.25cm,bottom=1.25cm}

% Change the font if you want to, depending on whether
% you're using pdflatex or xelatex/lualatex
\ifxetexorluatex
  % If using xelatex or lualatex:
  \setmainfont{Lato}
\else
  % If using pdflatex:
  \usepackage[utf8]{inputenc}
  \usepackage[T1]{fontenc}
  \usepackage[default]{lato}
\fi

% Change the colours if you want to
\definecolor{Mulberry}{HTML}{72243D}
\definecolor{SlateGrey}{HTML}{2E2E2E}
\definecolor{LightGrey}{HTML}{666666}
\colorlet{heading}{Sepia}
\colorlet{accent}{Mulberry}
\colorlet{emphasis}{SlateGrey}
\colorlet{body}{LightGrey}

% Change the bullets for itemize and rating marker
% for \cvskill if you want to
\renewcommand{\itemmarker}{{\small\textbullet}}
\renewcommand{\ratingmarker}{\faCircle}

%% sample.bib contains your publications
\addbibresource{sample.bib}

\begin{document}
\name{Letícia Pavesi Mansano}
\tagline{Technical Writer}
%\photo{2.8cm}{Globe_High}
\personalinfo{%
  % Not all of these are required!
  % You can add your own with \printinfo{symbol}{detail}
  \email{le.pavesi@gmail.com}
  \phone{+55 (41) 998 - 008 - 322}
  \mailaddress{Rua Guaianazes 1045, ap. 43}
  \location{Curitiba - PR, Brazil}
  
  \homepage{mansanocreations.wordpress.com}
  \linkedin{linkedin.com/in/leticiapavesimansano}
  \twitter{@le\_pavesi}
  \github{github.com/leticiapavesimansano}
  %% You MUST add the academicons option to \documentclass, then compile with LuaLaTeX or XeLaTeX, if you want to use \orcid or other academicons commands.
  % \orcid{orcid.org/0000-0000-0000-0000}
}

%% Make the header extend all the way to the right, if you want.
\begin{fullwidth}
\makecvheader
\end{fullwidth}

%% Depending on your tastes, you may want to make fonts of itemize environments slightly smaller
% \AtBeginEnvironment{itemize}{\small}

%% Provide the file name containing the sidebar contents as an optional parameter to \cvsection.
%% You can always just use \marginpar{...} if you do
%% not need to align the top of the contents to any
%% \cvsection title in the "main" bar.
\cvsection[page1sidebar]{Experience}

  \cvevent{Project Coordinator}{Pumatronix Equipamentos Eletrônicos Ltda.}{2018 -- Ongoing}{Curitiba, Brazil}
  \begin{itemize}
    \item Technical documentation requests and team manager
    \item Style Guide and document template design
    \item Portuguese and English documentation design using git (Bitbucket) and Markdown (i.e. manual, quickstart guide, Application Notes)
    \item Product specification development following ISO 9001:2008 (product requirements, release plans, and project report)
    \item Brazilian Informatics Law Report development
  \end{itemize}
  \divider

  \cvevent{Project and QA Coordinator }{Pumatronix Equipamentos Eletrônicos Ltda.}{2016 -- 2018}{Curitiba, Brazil}
  \begin{itemize}
    \item Portuguese and English documentation design using Microsoft Office (i.e. manual, quickstart guide, Application Notes)
    \item Product specification development following ISO 9001:2008 (product requirements, release plans, and project report)
    \item Brazilian Informatics Law Report development
    \item Scrum Master of the hardware development team
    \item Scrum Master of the QA (Quality Assurance) team
  \end{itemize}
  \divider

  \cvevent{Project Analyst}{Pumatronix Equipamentos Eletrônicos Ltda.}{2015 -- 2016}{Curitiba, Brazil}
  \begin{itemize}
    \item Portuguese and English documentation design using Microsoft Office (i.e. manual, quickstart guide, Application Notes)
    \item Product specification development following ISO 9001:2008 (product requirements, release plans, and project report)
    \item Brazilian Informatics Law Report development
    \item Scrum Master of the hardware development team and the software development team
  \end{itemize}
  \divider

  \cvevent{Support Analyst}{Pumatronix Equipamentos Eletrônicos Ltda.}{2014 -- 2015}{Curitiba, Brazil}
  \begin{itemize}
    \item Customer technical support for mobility systems and hardware
    \item Remote product configuration
    \item Product manuals development in Portuguese and English
    \item New ITSCAM Web Interface development
    \item SCRUM implementation on the development department
  \end{itemize}

\begin{fullwidth}
  \cvsection{Experience}
    \cvevent{Development Analyst}{Gaussian Inteligência Computacional Ltda.}{2013 -- 2014}{Curitiba, Brazil}
    \begin{itemize}
      \item Software development for embedded computer vision algorithms on the camera ITSCAM. The software identifies vehicles on day and night images and triggers OCR (Optical Character Recognition) on vehicle license plates
      \item Software development for computer vision validation algorithms
    \end{itemize}
    \divider

    \cvevent{Internship on Embedded Systems R\&D}{Hi Technologies}{2010 -- 2011}{Curitiba, Brazil}
    \begin{itemize}
      \item Software development for an Android Digital Oximeter
      \item Device driver programming to integrate embedded camera on Android (using Java and C/C++)
    \end{itemize}

%% Yeah I didn't spend too much time making all the 
%% spacing consistent... sorry. Use \smallskip, \medskip, 
%% \bigskip, \vpsace etc to make ajustments.

  \medskip

  \cvsection{Education}
    \cvevent{MBA\ in Your in Project Management}{OPET}{2014 -- 2016}{Curitiba, Brazil}
    \divider

    \cvevent{M.Sc.\ in Science}{Universidade Tecnológica Federal do Paraná}{2011 -- 2013}{Curitiba, Brazil}
    Title: Augmented Reality System for Aerial Images

    \divider

    \cvevent{B.Sc.\ in Computer Engineering}{Universidade Estadual de Ponta Grossa}{2006 -- 2010}{Ponta Grossa, Brazil}
    \divider

    \cvevent{Technologist\ in Industrial Automation}{Universidade Tecnológica Federal do Paraná}{2007 -- 2009}{Ponta Grossa, Brazil}
\end{fullwidth}

\nocite{*}

%%\clearpage
%%\cvsection[page2sidebar]{Publications}

%% If the NEXT page doesn't start with a \cvsection but you'd
%% still like to add a sidebar, then use this command on THIS
%% page to add it. The optional argument lets you pull up the
%% sidebar a bit so that it looks aligned with the top of the
%% main column.
% \addnextpagesidebar[-1ex]{page3sidebar}

\end{document}
